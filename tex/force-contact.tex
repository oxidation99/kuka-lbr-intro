\documentclass{beamer}
\usepackage{presentation}

\begin{document}
\section{Force Contact}

\begin{frame}[fragile]{Initialization}
  \metroset{block=fill}
  \begin{block}{DemoForceContactRetract.java}
    \inputminted[
    breaklines,
    linenos,
    fontsize=\tiny,
    firstline=32,
    lastline=41
    ]{java}{./code/DemoForceContactRetract.java}
  \end{block}

  \begin{block}{Details}
    \begin{itemize}
      \item Initialize the robot, controller, and tool objects
    \end{itemize}
  \end{block}
\end{frame}

\begin{frame}[fragile]{Application Start}
  \metroset{block=fill}
  \begin{block}{DemoForceContactRetract.java}
    \inputminted[
    breaklines,
    linenos,
    fontsize=\tiny,
    firstline=43,
    lastline=51
    ]{java}{./code/DemoForceContactRetract.java}
  \end{block}

  \begin{block}{Details}
    \begin{itemize}
      \item Tool is loaded onto robot object
      \item Robot is moved to a start position in joint-space
    \end{itemize}
  \end{block}
\end{frame}

\begin{frame}[fragile]{Motion Setup}
  \metroset{block=fill}
  \begin{block}{DemoForceContactRetract.java}
    \inputminted[
    breaklines,
    linenos,
    fontsize=\tiny,
    firstline=53,
    lastline=69
    ]{java}{./code/DemoForceContactRetract.java}
  \end{block}
\end{frame}

\begin{frame}{Motion Setup}
  \metroset{block=fill}
  \begin{block}{Details}
    \begin{itemize}
      \item Define start and end frames
      \item Define force conditions
      \begin{itemize}
        \item Position and orientation of force condition?
      \end{itemize}
      \item Define motions objects
      \begin{itemize}
        \item Reusable
      \end{itemize}
    \end{itemize}
  \end{block}
\end{frame}

\begin{frame}[fragile]{Motion Loop}
  \metroset{block=fill}
  \begin{block}{DemoForceContactRetract.java}
    \inputminted[
    breaklines,
    linenos,
    fontsize=\tiny,
    firstline=71,
    lastline=78
    ]{java}{./code/DemoForceContactRetract.java}
  \end{block}

  \begin{block}{Details}
    \begin{itemize}
      \item Loop previously defined motions
      \item Synchronous move commands
    \end{itemize}
  \end{block}
\end{frame}

\begin{frame}[fragile]{Move Function}
  \metroset{block=fill}
  \begin{block}{DemoForceContactRetract.java}
    \inputminted[
    breaklines,
    linenos,
    fontsize=\tiny,
    firstline=50,
    lastline=51
    ]{java}{./code/DemoForceContactRetract.java}
  \end{block}

  \begin{block}{DemoForceContactRetract.java}
    \inputminted[
    breaklines,
    linenos,
    fontsize=\tiny,
    firstline=73,
    lastline=74
    ]{java}{./code/DemoForceContactRetract.java}
  \end{block}

  \begin{block}{Details}
    \begin{itemize}
      \item What's the difference between these two code blocks?
      \item Are these move commands equivalent? In joint-space? In Cartesian-space?
    \end{itemize}
  \end{block}
\end{frame}
\end{document}
