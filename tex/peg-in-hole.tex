\documentclass{beamer}
\usepackage{presentation}

\begin{document}
\section{Peg-in-hole}

\begin{frame}{Servo Motion}
  \metroset{block=fill}
  \begin{block}{DemoPegInHole.java}
    \inputminted[
    breaklines,
    linenos,
    fontsize=\tiny,
    firstline=83,
    lastline=93
    ]{java}{./code/DemoPegInHole.java}
  \end{block}

  \begin{block}{Details}
    \begin{itemize}
      \item Advanced real-time control
      \item Adjust motion on-the-fly
      \item Event-loops
    \end{itemize}
  \end{block}
\end{frame}

\begin{frame}{Search for Force}
  \metroset{block=fill}
  \begin{block}{DemoPegInHole.java}
    \inputminted[
    breaklines,
    linenos,
    fontsize=\tiny,
    firstline=108,
    lastline=113
    ]{java}{./code/DemoPegInHole.java}
  \end{block}

  \begin{block}{Details}
    \begin{itemize}
      \item Look for desired force vector in XY plane
    \end{itemize}
  \end{block}
\end{frame}

\begin{frame}{Sinusoidal Motion}
  \metroset{block=fill}
  \begin{block}{DemoPegInHole.java}
    \inputminted[
    breaklines,
    linenos,
    fontsize=\tiny,
    firstline=115,
    lastline=125
    ]{java}{./code/DemoPegInHole.java}
  \end{block}

  \begin{block}{Details}
    \begin{itemize}
      \item Search for collision using sinusoidal movement
    \end{itemize}
  \end{block}
\end{frame}

\begin{frame}{Sinusoidal Motion}
  \metroset{block=fill}
  \begin{block}{DemoPegInHole.java}
    \inputminted[
    breaklines,
    linenos,
    fontsize=\tiny,
    firstline=115,
    lastline=125
    ]{java}{./code/DemoPegInHole.java}
  \end{block}

  \begin{block}{Details}
    \begin{itemize}
      \item Search for collision using sinusoidal movement
    \end{itemize}
  \end{block}
\end{frame}
\end{document}
