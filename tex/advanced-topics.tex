\documentclass{beamer}
\usepackage{presentation}

\begin{document}
\section{Advanced Topics}

\begin{frame}{Data Acquisition}
  \metroset{block=fill}
  \begin{block}{PHDDaqPokeAutomated.java}
    \inputminted[
    breaklines,
    linenos,
    fontsize=\tiny,
    firstline=52,
    lastline=66
    ]{java}{./code/PHDDaqPokeAutomated.java}
  \end{block}
\end{frame}

\begin{frame}{Data Acquisition}
  \metroset{block=fill}
  \begin{block}{Details}
    \begin{itemize}
      \item \SI{1000}{\hertz} data recording
      \item A lot of robot data is available
      \item Can be for machine learning, logging, etc.
    \end{itemize}
  \end{block}
\end{frame}

\begin{frame}{Multithreading}
  \metroset{block=fill}
  \begin{block}{DAQForceContactRetract.java}
    \inputminted[
    breaklines,
    linenos,
    fontsize=\tiny,
    firstline=51,
    lastline=61
    ]{java}{./code/DAQForceContactRetract.java}
  \end{block}
\end{frame}

\begin{frame}{Multithreading}
  \metroset{block=fill}
  \begin{block}{DAQForceContactRetract.java}
    \inputminted[
    breaklines,
    linenos,
    fontsize=\tiny,
    firstline=88,
    lastline=97
    ]{java}{./code/DAQForceContactRetract.java}
  \end{block}

  \begin{block}{Details}
    \begin{itemize}
      \item Take advantage of the \textit{java.util.concurrent} library for advanced functionality
    \end{itemize}
  \end{block}
\end{frame}

\begin{frame}{FRI}
  \metroset{block=fill}
  \begin{itemize}
    \item Real-time server-client communications
    \item Allows external joint, position, and torque control
    \item Java API for FRI server on Sunrise controller
    \item C++ or Java API for FRI client on external PC
    \item Real-time client OS recommended
  \end{itemize}
\end{frame}

\begin{frame}{Server-client Communications}
  \metroset{block=fill}
  \begin{block}{ServerInitializer.java}
    \inputminted[
    breaklines,
    linenos,
    fontsize=\tiny,
    firstline=26,
    lastline=40
    ]{java}{./code/ServerInitializer.java}
  \end{block}

  \begin{block}{Details}
    \begin{itemize}
      \item Use 3rd-party libraries such as \textit{Netty}
      \item \small\url{github.com/nnadeau/robot-communication-java}
    \end{itemize}
  \end{block}
\end{frame}

\begin{frame}{Protobuf}
  \metroset{block=fill}
  \begin{block}{robot\_packet.proto}
    \inputminted[
    breaklines,
    linenos,
    fontsize=\tiny,
    firstline=9,
    lastline=20
    ]{yaml}{./code/robot_packet.proto}
  \end{block}

  \begin{block}{Details}
    \begin{itemize}
      \item Language-neutral protocol definitions
      \item Auto-generates packet classes for all languages (e.g., Python, Java, C++)
      \item \small\url{github.com/nnadeau/robot-communication-packet-proto}
    \end{itemize}
  \end{block}
\end{frame}

\begin{frame}{External Libraries}
  \metroset{block=fill}
  \begin{block}{build.gradle}
    \inputminted[
    breaklines,
    linenos,
    fontsize=\tiny,
    firstline=17,
    lastline=27
    ]{groovy}{./code/build.gradle}
  \end{block}

  \begin{block}{Details}
    \begin{itemize}
      \item Build external packages targeting JDK 1.6
      \item Import libraries from \textit{Maven}
      \item Use build tools such as \textit{Gradle}
    \end{itemize}
  \end{block}
\end{frame}

\begin{frame}{State Machines}
  \metroset{block=fill}
  \begin{block}{ESMState.java}
    \inputminted[
    breaklines,
    linenos,
    fontsize=\tiny,
    firstline=3,
    lastline=11
    ]{java}{./code/ESMState.java}
  \end{block}

  \begin{block}{Details}
    \begin{itemize}
      \item Safety states can be context-based
      \item Finite state machine (FSM) design can be useful for a multi-context application, e.g.:
      \begin{enumerate}
        \item Non-collaborative material handling motions
        \item Collaborative assembly
        \item Non-collaborative material handling motions
      \end{enumerate}
    \end{itemize}
  \end{block}
\end{frame}
\end{document}
